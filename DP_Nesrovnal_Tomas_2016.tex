% options:
% thesis=B bachelor's thesis
% thesis=M master's thesis
% czech thesis in Czech language
% slovak thesis in Slovak language
% english thesis in English language
% hidelinks remove colour boxes around hyperlinks

\documentclass[thesis=M,czech]{FITthesis}[2012/06/26]

\usepackage[utf8]{inputenc} % LaTeX source encoded as UTF-8

\usepackage{graphicx} %graphics files inclusion
% \usepackage{amsmath} %advanced maths
% \usepackage{amssymb} %additional math symbols

\usepackage{dirtree} %directory tree visualisation

% % list of acronyms
% \usepackage[acronym,nonumberlist,toc,numberedsection=autolabel]{glossaries}
% \iflanguage{czech}{\renewcommand*{\acronymname}{Seznam pou{\v z}it{\' y}ch zkratek}}{}
% \makeglossaries

\newcommand{\tg}{\mathop{\mathrm{tg}}} %cesky tangens
\newcommand{\cotg}{\mathop{\mathrm{cotg}}} %cesky cotangens

% % % % % % % % % % % % % % % % % % % % % % % % % % % % % % 
% ODTUD DAL VSE ZMENTE
% % % % % % % % % % % % % % % % % % % % % % % % % % % % % % 

\department{Katedra teoretické informatiky}
\title{Comma-shell, interaktivní debugger shellu}
\authorGN{Tomáš} %(křestní) jméno (jména) autora
\authorFN{Nesrovnal} %příjmení autora
\authorWithDegrees{Bc. Tomáš Nesrovnal} %jméno autora včetně současných akademických titulů
\supervisor{Ing. Jan Baier}
\acknowledgements{Doplňte, máte-li komu a za co děkovat. V~opačném případě úplně odstraňte tento příkaz.}
\abstractCS{V~několika větách shrňte obsah a přínos této práce v~češtině. Po přečtení abstraktu by se čtenář měl mít čtenář dost informací pro rozhodnutí, zda chce Vaši práci číst.}
\abstractEN{Sem doplňte ekvivalent abstraktu Vaší práce v~angličtině.}
\placeForDeclarationOfAuthenticity{V~Praze}
\declarationOfAuthenticityOption{4} %volba Prohlášení (číslo 1-6)
\keywordsCS{Nahraďte seznamem klíčových slov v češtině oddělených čárkou.}
\keywordsEN{Nahraďte seznamem klíčových slov v angličtině oddělených čárkou.}
\website{https://github.com/nesro/nesrotom-dip-2016} %volitelná URL práce, objeví se v tiráži - úplně odstraňte, nemáte-li URL práce

\begin{document}

% \newacronym{CVUT}{{\v C}VUT}{{\v C}esk{\' e} vysok{\' e} u{\v c}en{\' i} technick{\' e} v Praze}
% \newacronym{FIT}{FIT}{Fakulta informa{\v c}n{\' i}ch technologi{\' i}}

\begin{introduction}

Grafické uživatelské rozhraní (GUI) se jednoduše ovládá, ale ne vždy je k dispozici. To platí zejména při ovládání serverů.

Rozhraní příkazové řádky (CLI) je základní textové prostředí pro komunikaci s operačním systémem. Umožňuje spouštění programů, vkládat vstupní data a sledovat výstupní data v terminálu.

Jedním ze základních bodů UNIXové filosofie je mít jednoduché programy, které dělají pouze jednu věc, ale dělají ji dobře. To platí zejména pro základní příkazy ze sady GNU coreutils, tedy příkazy pro základní manipulaci se soubory, shellem a textem.

Tyto základní příkazy je možné řetězit a tím vytvářet užitečné jednořádkové skripty.

TODO: napsat o tom, ze pro zacatecnika to muze byt neintuitivni, musi si pamatovat spoustu prikazu. o navratovych kodech, o historii prikazu a o logovani

% https://www.gnu.org/software/coreutils/coreutils.html
% http://www.tldp.org/LDP/GNU-Linux-Tools-Summary/GNU-Linux-Tools-Summary.pdf

% https://en.wikipedia.org/wiki/Unix_philosophy
% ^- tady je spousta dalsich uzitecnych odkazu, o cem by se mohlo psat

% https://en.wikipedia.org/wiki/Command-line_interface
% https://en.wikipedia.org/wiki/Shell_(computing)
% https://www.abclinuxu.cz/ucebnice/zaklady/prikazova-radka

\end{introduction}

\chapter{Cíl práce}
Napsat o tom, ze cilem prace je usnadnit praci v prikazove radce a sepsat zakladni funkcionalitu debuggeru.


\chapter{Analýza a návrh}



\section{Historie UNIXu}

% unix timeline for students
% http://unix.harley.com/instructors/timeline.html
% http://www.harley.com/books/sg3.html

% Recommended UNIX Books
% http://www.it.northwestern.edu/research/user-services/sscc/booklist.html

\section{Historie shellu}

\begin{figure}[htb]\centering
	\includegraphics[width=\textwidth]{./images/tmp_shell_history}
	\caption{Historie Shellu}
	\label{fig:shell_history}
\end{figure}

V latexovych komentarich jsou nejake odkazy o historii.

% uvod, prehled csh, ksh, bash.
% https://www.ibm.com/developerworks/library/l-linux-shells/

% clanek od bourna z 1983, uvod do shellu, drtiva vetsina plati i dneska
%https://ia800300.us.archive.org/1/items/byte-magazine-1983-10/1983_10_BYTE_08-10_UNIX.pdf

% puvodni shell z unix v6
%https://github.com/JNeitzel/v6shell



%^ Historie shellu %%%%%%%%%%%%%%%%%%%%%%%%%%%%%%%%%%%%%%%%%%%%%%%%%%%%%%%%%%%%




\section{Fungování shellu}

\subsection{Gramatika shellu}
Soubor s gramatikiou bashe, parse.y, ma pres 6000 radek. Chtel bych zde napsat zjednodusenou gramatiku, ktera by se dala snadno pochopit (ono to zas tak slozite neni).

Popsat jakym zpusobem parsuje gramatiku BASH (yacc) a jakym to delaji parsery bashlex a bashast.


% https://github.com/idank/bashlex/blob/master/bashlex/parser.py
% https://github.com/neloe/libbash/blob/master/bashast/bashast.g

% https://stackoverflow.com/questions/5491775/how-to-write-a-shell-lexer-by-hand

\subsection{Spouštění příkazů}
Popsat základní principy jak funguje shell. Popsat procesy v unixu, fork, exec, co vsechno se musi stat, aby shell mohl spustit prikaz.

\subsection{Struktura BASHe}
Zdrojovy kod je rozdelen do souboru, mozna by bylo dobre popsat popsat co ktery soubor dela, aby si ctenar udelal.

% bash git repo
% git clone git://git.savannah.gnu.org/bash.git

% buffering in standard streams
% http://www.pixelbeat.org/programming/stdio_buffering/

\section{Debugovaní shellu} %-------------------------------------------------

\subsection{Debugovaní BASHe}
Jeste jsem se k tomu nedostal, ale chtel bych si odkrokovat v gdb par zakladnich veci, abych videl co se tam deje. Mozna by stalo za to septat i sem.

\subsection{Interní nástroje}

\subsubsection{Debugovací mód BASHe (jak funguje shopt s extdebug)}
shopt s extdebug

\subsubsection{set x, u, v, e}
priklady do skriptu

\subsubsection{PS0, PS4}
PS0 bude v novém bashi, my ji proto nebudeme používat, PS4 se vypisuje při debugovaní

\subsection{Externí nástroje}

\subsection{BASH Debugger}
todo: popsat jak funguje, co vsechno umi, nejake priklady
% http://bashdb.sourceforge.net/

\section{Možnosti debugování v interaktivním shellu} %------------------------

\subsection{GNU Readline}
GNU Readline umožnuje přemapovat enter tak, abysme mohli spustit prikaz v nami definovane funkci. Problémem je, že takto upravený příkaz se uloží do historie. Dalším problémem jsou víceřádkové příkazy, tedy takové, pro jejichž napsání musíme několikrát zmáčknout enter.
TODO: ukázka.

\subsection{Napsání nového REPLu}
Zprovoznění základní funkcionality by bylo snadné, vzhledem ke komplexnosti BASHe však téměř nemožné mít stejné chování jako v BASHi.

\subsection{DEBUG trap}
Současné řešení. Při zapnutém extdebug je možné příkazy nepustit a jen evalovat poslední příkaz z historie. TODO: je potřeba popsat základní chování historie (např mezera na začátku příkazu, atd.)


\section{Logování výstupu}
Napsat o tom, jak a proc logovat vystup skriptu. Jak to delat ve skriptech, jak to delat v interaktivnim shellu. Moznosti zapinani debugovani ve skriptech. Popsat prikaz jak prikaz script, tak i logovani pres exec. Popsat co se tam vsechno deje.




\section{Automatizované spouštění příkazů}
Popsat jak se dají automaticky spouštět příkazy. At uz lokalne, nebo vzdalene. Popsat jak rekonstrukci z typescriptu, tak treba Tcl, Expect.

\subsection{Testování shell skriptů}
Popsat jak funguji nektere testovaci frameworky.shunit2, roundup

% https://bmizerany.github.io/roundup/

\section{Požadavky na interaktivní debugger} %------------------------
Zde by mohly byt sepsany obecne pozadavky na to, co by interaktivni debugger mel vlastne delat.

\section{Chování uživatelů v příkazové řádce} %------------------------
Napadlo me udelat aketu o tom, jak se uzivatele chovaji v prikazove radce. Napriklad jaky pouzivaji shell, jaky maji PS1, nastavenou historii, logovani atd, jake pouzivaji aliasy, jestli si delaji skripty na kazdou vec, atd.

Asi by stalo za to vysledovat chovani studentu predmetu PS1 a pomoci jim v prevenci chyb ktere delaji.

% https://www.reddit.com/r/programming/comments/697cu/bash_users_what_do_you_have_for_your_ps1/


%%%%%%%%%%%%%%%%%%%%%%%%%%%%%%%%%%%%%%%%%%%%%%%%%%%%%%%%%%%%%%%%%%%%%%%%%%%%%%%
\chapter{Realizace}

\section{Nespouštění příkazů}
Pro zabránění spouštění používáme DEBUG trap.

\section{Hooks}
Aby byl kód přehledný, veškerá funkcionalita je rozdělena do hooků, nebo-li modulů, které obsahují kód, který je spuštěn před, nebo i po vykonání příkazu. Kód vykonaný před příkazem může rozhodnout, zda-li má dojít k zabránění vykonání příkazu.

\section{Bezpečný mód}
Bezpečný mód umožňuje dvě základní věci. Tou jednodušší je pouze vypsání efektu příkazu, který má nějaké destruktivní. Složitější varianta dovoluje vracení do stavu před vykonáním příkazu. (TODO: tohle ještě není naimplementované)

\section{Historie}
Popsat jak jsem vyresil ukladani historie prikazu, ukladani vystupu, navratove kody, jak vracet nasledky prikazu do puvodniho stavu.




\begin{conclusion}
	sem napište závěr Vaší práce
\end{conclusion}

\bibliographystyle{csn690}
\bibliography{mybibliographyfile}

\appendix

\chapter{Seznam použitých zkratek}
% \printglossaries
\begin{description}
	\item[GUI] Graphical user interface
	\item[XML] Extensible markup language
\end{description}


% % % % % % % % % % % % % % % % % % % % % % % % % % % % 
% % Tuto kapitolu z výsledné práce ODSTRAŇTE.
% % % % % % % % % % % % % % % % % % % % % % % % % % % % 
% 
% \chapter{Návod k~použití této šablony}
% 
% Tento dokument slouží jako základ pro napsání závěrečné práce na Fakultě informačních technologií ČVUT v~Praze.
% 
% \section{Výběr základu}
% 
% Vyberte si šablonu podle druhu práce (bakalářská, diplomová), jazyka (čeština, angličtina) a kódování (ASCII, \mbox{UTF-8}, \mbox{ISO-8859-2} neboli latin2 a nebo \mbox{Windows-1250}). 
% 
% V~české variantě naleznete šablony v~souborech pojmenovaných ve formátu práce\_kódování.tex. Typ může být:
% \begin{description}
% 	\item[BP] bakalářská práce,
% 	\item[DP] diplomová (magisterská) práce.
% \end{description}
% Kódování, ve kterém chcete psát, může být:
% \begin{description}
% 	\item[UTF-8] kódování Unicode,
% 	\item[ISO-8859-2] latin2,
% 	\item[Windows-1250] znaková sada 1250 Windows.
% \end{description}
% V~případě nejistoty ohledně kódování doporučujeme následující postup:
% \begin{enumerate}
% 	\item Otevřete šablony pro kódování UTF-8 v~editoru prostého textu, který chcete pro psaní práce použít -- pokud můžete texty s~diakritikou normálně přečíst, použijte tuto šablonu.
% 	\item V~opačném případě postupujte dále podle toho, jaký operační systém používáte:
% 	\begin{itemize}
% 		\item v~případě Windows použijte šablonu pro kódování \mbox{Windows-1250},
% 		\item jinak zkuste použít šablonu pro kódování \mbox{ISO-8859-2}.
% 	\end{itemize}
% \end{enumerate}
% 
% 
% V~anglické variantě jsou šablony pojmenované podle typu práce, možnosti jsou:
% \begin{description}
% 	\item[bachelors] bakalářská práce,
% 	\item[masters] diplomová (magisterská) práce.
% \end{description}
% 
% \section{Použití šablony}
% 
% Šablona je určena pro zpracování systémem \LaTeXe{}. Text je možné psát v~textovém editoru jako prostý text, lze však také využít specializovaný editor pro \LaTeX{}, např. Kile.
% 
% Pro získání tisknutelného výstupu z~takto vytvořeného souboru použijte příkaz \verb|pdflatex|, kterému předáte cestu k~souboru jako parametr. Vhodný editor pro \LaTeX{} toto udělá za Vás. \verb|pdfcslatex| ani \verb|cslatex| \emph{nebudou} s~těmito šablonami fungovat.
% 
% Více informací o~použití systému \LaTeX{} najdete např. v~\cite{wikilatex}.
% 
% \subsection{Typografie}
% 
% Při psaní dodržujte typografické konvence zvoleného jazyka. České \uv{uvozovky} zapisujte použitím příkazu \verb|\uv|, kterému v~parametru předáte text, jenž má být v~uvozovkách. Anglické otevírací uvozovky se v~\LaTeX{}u zadávají jako dva zpětné apostrofy, uzavírací uvozovky jako dva apostrofy. Často chybně uváděný symbol "{} (palce) nemá s~uvozovkami nic společného.
% 
% Dále je třeba zabránit zalomení řádky mezi některými slovy, v~češtině např. za jednopísmennými předložkami a spojkami (vyjma \uv{a}). To docílíte vložením pružné nezalomitelné mezery -- znakem \texttt{\textasciitilde}. V~tomto případě to není třeba dělat ručně, lze použít program \verb|vlna|.
% 
% Více o~typografii viz \cite{kobltypo}.
% 
% \subsection{Obrázky}
% 
% Pro umožnění vkládání obrázků je vhodné použít balíček \verb|graphicx|, samotné vložení se provede příkazem \verb|\includegraphics|. Takto je možné vkládat obrázky ve formátu PDF, PNG a JPEG jestliže používáte pdf\LaTeX{} nebo ve formátu EPS jestliže používáte \LaTeX{}. Doporučujeme preferovat vektorové obrázky před rastrovými (vyjma fotografií).
% 
% \subsubsection{Získání vhodného formátu}
% 
% Pro získání vektorových formátů PDF nebo EPS z~jiných lze použít některý z~vektorových grafických editorů. Pro převod rastrového obrázku na vektorový lze použít rasterizaci, kterou mnohé editory zvládají (např. Inkscape). Pro konverze lze použít též nástroje pro dávkové zpracování běžně dodávané s~\LaTeX{}em, např. \verb|epstopdf|.
% 
% \subsubsection{Plovoucí prostředí}
% 
% Příkazem \verb|\includegraphics| lze obrázky vkládat přímo, doporučujeme však použít plovoucí prostředí, konkrétně \verb|figure|. Například obrázek \ref{fig:float} byl vložen tímto způsobem. Vůbec přitom nevadí, když je obrázek umístěn jinde, než bylo původně zamýšleno -- je tomu tak hlavně kvůli dodržení typografických konvencí. Namísto vynucování konkrétní pozice obrázku doporučujeme používat odkazování z~textu (dvojice příkazů \verb|\label| a \verb|\ref|).
% 
% \begin{figure}\centering
% 	\includegraphics[width=0.5\textwidth, angle=30]{cvut-logo-bw}
% 	\caption[Příklad obrázku]{Ukázkový obrázek v~plovoucím prostředí}\label{fig:float}
% \end{figure}
% 
% \subsubsection{Verze obrázků}
% 
% % Gnuplot BW i barevně
% Může se hodit mít více verzí stejného obrázku, např. pro barevný či černobílý tisk a nebo pro prezentaci. S~pomocí některých nástrojů na generování grafiky je to snadné.
% 
% Máte-li například graf vytvořený v programu Gnuplot, můžete jeho černobílou variantu (viz obr. \ref{fig:gnuplot-bw}) vytvořit parametrem \verb|monochrome dashed| příkazu \verb|set term|. Barevnou variantu (viz obr. \ref{fig:gnuplot-col}) vhodnou na prezentace lze vytvořit parametrem \verb|colour solid|.
% 
% \begin{figure}\centering
% 	\includegraphics{gnuplot-bw}
% 	\caption{Černobílá varianta obrázku generovaného programem Gnuplot}\label{fig:gnuplot-bw}
% \end{figure}
% 
% \begin{figure}\centering
% 	\includegraphics{gnuplot-col}
% 	\caption{Barevná varianta obrázku generovaného programem Gnuplot}\label{fig:gnuplot-col}
% \end{figure}
% 
% 
% \subsection{Tabulky}
% 
% Tabulky lze zadávat různě, např. v~prostředí \verb|tabular|, avšak pro jejich vkládání platí to samé, co pro obrázky -- použijte plovoucí prostředí, v~tomto případě \verb|table|. Například tabulka \ref{tab:matematika} byla vložena tímto způsobem.
% 
% \begin{table}\centering
% 	\caption[Příklad tabulky]{Zadávání matematiky}\label{tab:matematika}
% 	\begin{tabular}{|l|l|c|c|}\hline
% 		Typ		& Prostředí		& \LaTeX{}ovská zkratka	& \TeX{}ovská zkratka	\tabularnewline \hline \hline
% 		Text		& \verb|math|		& \verb|\(...\)|	& \verb|$...$|		\tabularnewline \hline
% 		Displayed	& \verb|displaymath|	& \verb|\[...\]|	& \verb|$$...$$|	\tabularnewline \hline
% 	\end{tabular}
% \end{table}
% 
% % % % % % % % % % % % % % % % % % % % % % % % % % % % 

\chapter{Obsah přiloženého CD}

%upravte podle skutecnosti

\begin{figure}
	\dirtree{%
		.1 readme.txt\DTcomment{stručný popis obsahu CD}.
		.1 exe\DTcomment{adresář se spustitelnou formou implementace}.
		.1 src.
		.2 impl\DTcomment{zdrojové kódy implementace}.
		.2 thesis\DTcomment{zdrojová forma práce ve formátu \LaTeX{}}.
		.1 text\DTcomment{text práce}.
		.2 thesis.pdf\DTcomment{text práce ve formátu PDF}.
		.2 thesis.ps\DTcomment{text práce ve formátu PS}.
	}
\end{figure}

\end{document}
